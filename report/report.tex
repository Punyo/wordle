\documentclass[a4j]{ujarticle}

\usepackage[dvipdfmx]{graphicx}
\usepackage{graphicx}
\usepackage{url}
\usepackage{listings}
\usepackage{ascmac}
\usepackage{amsmath,amssymb}

\lstset{
  basicstyle={\ttfamily},
  identifierstyle={\small},
  commentstyle={\small\itshape},
  keywordstyle={\small\bfseries},
  ndkeywordstyle={\small},
  stringstyle={\small\ttfamily},
  frame={tb},
  breaklines=true,
  columns=[l]{fullflexible},
  numbers=left,
  xrightmargin=0zw,
  xleftmargin=3zw,
  numberstyle={\scriptsize},
  stepnumber=1,
  numbersep=1zw,
  lineskip=-0.5ex
}

\title{タイトル}
\date{\today}
\author{\input{./authors.txt}}
\begin{document}
\maketitle
\section{概要}
\section{プロジェクトの進め方}
プロジェクトを進めるにあたり、以下のように実装部分の役割分担を行った。
\begin{description}
  \item[\input{./authorNameShort1.txt}:]\texttt{com.programming.advanced.wordle.service}、\texttt{com.programming.advanced.wordle.model}の実装やテスト、一部の\texttt{GameController}の実装を担当。
  \item[\input{./authorNameShort2.txt}:]
  \item[\input{./authorNameShort3.txt}:] 
\end{description}
バージョン管理システムにはGitを使用し、GitHub上でリポジトリのホスティングを行い共同制作を行った。リポジトリのURLを以下に示す。
\begin{center}
  \url{https://github.com/Punyo/wordle}
\end{center}
\section{機能}
\section{実装}
\subsection{com.programming.advanced.wordle.service}
このパッケージには主にゲームの進行管理や判定ロジック等が実装されている。以下のクラスが含まれる。
\begin{itemize}
  \item \texttt{GameService}:ゲームの状態(正解単語、残り試行回数、単語長など)を管理し、ゲームの開始や入力単語の判定処理を行うSingletonであるクラスである。
  \item \texttt{WordBoxStatus}:\texttt{GameService.checkWord(String inputWord)}での判定結果を表す列挙型。
  \item \texttt{GameServiceTest}:\texttt{GameService}のテストクラスで、JUnitを使用してゲームのロジックが正しく動作するかを確認する。
  \item \texttt{GameServiceNotInitializedTest}:\texttt{GameService}が初期化されていない状態でのテストクラスで、ゲームの初期化が行われていない場合の挙動を確認する。
\end{itemize}

それぞれのクラスの詳細を以下に示す。なお、getter/setterは省略する。
\subsubsection{GameService.java}

\begin{itemize}
  \item \texttt{getInstance()}:\texttt{GameService}のインスタンスを取得する。
  \item \texttt{startNewGame(String word, int attempts, int wordLength)}:新しいゲームを開始し、正解単語・試行回数・単語長をセットし、初期化フラグを立てる。
  \item \texttt{checkWord(String inputWord)}:入力単語を正解単語と比較し、各文字ごとに\texttt{WordBoxStatus}(正解・含まれる・含まれない)を判定して返す。試行回数も減少する。\\
  なお、引数の単語が正解の単語の長さと異なる場合、DBからのデータの読み取りに失敗した場合、試行回数が残っていない場合、\texttt{startNewGame} での初期化が行われていなかった場合は例外を投げる。さらに、入力語がDBに登録されていない場合はnullを返す。
\end{itemize}

\subsubsection{WordBoxStatus.java}

\begin{itemize}
  \item \texttt{CORRECT}:文字が正しい位置にある場合。
  \item \texttt{NOT\_IN\_WORD}:文字が単語に含まれない場合。
  \item \texttt{IN\_WORD}:文字が単語に含まれるが位置が違う場合。
\end{itemize}

\subsubsection{GameServiceTest.java}

\begin{itemize}
  \item \texttt{testCheckWordLength()}:入力単語の長さが正解単語と異なる場合に\texttt{IllegalArgumentException}が投げられることを確認するテスト。
  \item \texttt{testCheckWordCorrect()}:正解単語を入力した場合、すべての判定が\texttt{CORRECT}となることを確認するテスト。
  \item \texttt{testCheckWordStatus()}:一部の文字が正解、一部が不正解の場合の判定結果(\texttt{CORRECT}、\texttt{NOT\_IN\_WORD}、\texttt{IN\_WORD})が正しいことを確認するテスト。
  \item \texttt{testCheckWordNoRemainingAttempts()}:試行回数が0になった後に入力した場合、\texttt{IllegalStateException}が投げられることを確認するテスト。
  \item \texttt{testGetters()}:\texttt{GameService}のgetterメソッドが正しく値を返すことを確認するテスト。
\end{itemize}

\subsubsection{GameServiceNotInitializedTest.java}

\begin{itemize}
  \item \texttt{testCheckWordNotInitialized()}:\texttt{GameService}が初期化されていない状態で\texttt{checkWord}を呼び出した場合、\texttt{IllegalStateException}が投げられることを確認するテスト。
\end{itemize}

\end{document}