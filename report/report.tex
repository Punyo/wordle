\documentclass[a4j]{ujarticle}

\usepackage[dvipdfmx]{graphicx}
\usepackage{graphicx}
\usepackage{url}
\usepackage{listings}
\usepackage{ascmac}
\usepackage{amsmath,amssymb}

\lstset{
  basicstyle={\ttfamily},
  identifierstyle={\small},
  commentstyle={\small\itshape},
  keywordstyle={\small\bfseries},
  ndkeywordstyle={\small},
  stringstyle={\small\ttfamily},
  frame={tb},
  breaklines=true,
  columns=[l]{fullflexible},
  numbers=left,
  xrightmargin=0zw,
  xleftmargin=3zw,
  numberstyle={\scriptsize},
  stepnumber=1,
  numbersep=1zw,
  lineskip=-0.5ex
}

\title{タイトル}
\date{\today}
\author{\input{./authors.txt}}
\begin{document}
\maketitle
\section{概要}
本プロジェクトは、日本語の5文字単語を用いたWordle風パズルゲームである。ユーザーは制限回数内で正解の単語を推測し当てることを目指す。視覚的および操作的な体験(UI/UX)は、本家Wordle(\url{https://wordle.mottox2.com})に準拠している。また、ゲーム開始・履歴表示・終了といった簡易メニュー画面の実装も行った。
\section{プロジェクトの進め方}
本プロジェクトを開始するにあたり、まずは要件定義を行い、必要な機能と非機能要件を明確にした。その後、各自の担当部分を決めて実装を進めた。
\subsection{要件}
\subsubsection{機能要件}
\begin{itemize}
  \item プレイごとにデータベースからランダムで1語を選出
  \item ユーザーによる5文字入力受付
  \item 入力語と正解語の一致度判定
  \begin{itemize}
    \item 正しい文字かつ正しい位置:緑
    \item 正しい文字だが位置が違う:黄
    \item 含まれていない文字:灰
  \end{itemize}
  \item 最大6回まで入力可能(失敗時はGame Over)
  \item 勝敗結果と履歴はデータベースに記録
\end{itemize}

\subsubsection{追加機能}
\begin{itemize}
  \item メニュー画面(新規プレイ、履歴確認、終了)
  \item 履歴表示(過去のゲームデータ)
\end{itemize}

\subsubsection{非機能要件}
\begin{itemize}
  \item データ永続化のためSQLiteを使用
  \item 単体テストしやすい構造(デザインパターン活用)
  \item 日本語入力対応(IME)
\end{itemize}
\subsection{役割分担}
\begin{description}
  \item[\input{./authorNameShort1.txt}:]\texttt{com.programming.advanced.wordle.service}、\texttt{com.programming.advanced.wordle.model}の実装やテスト、一部の\texttt{GameController}の実装を担当。
  \item[\input{./authorNameShort2.txt}:]\texttt{com.programming.advanced.wordle.dao}におけるDB周りの実装やテスト、DB作成時に利用するリソース(\path{wordle/src/main/resources})の作成を担当。
  \item[\input{./authorNameShort3.txt}:]\texttt{com.programming.advanced.wordle.controller}の実装、JavaFX を用いた GUIの構築、ローマ字をひらがなに変換する処理の実装を担当。
\end{description}
\subsection{開発環境}
本プロジェクトはJavaFXを使用してGUIの構築を行った。エディタにはVSCode、Eclipse、IntelliJ IDEAを使用し、ビルドツールにはMavenを採用した。データベースにはSQLite3を使用し、テストフレームワークにはJUnitを使用した。FXMLファイルの作成には、JavaFX Scene Builderを使用した。\\
バージョン管理システムにはGitを使用し、GitHub上でリポジトリのホスティングを行い共同制作を行った。リポジトリのURLを以下に示す。
\begin{center}
  \url{https://github.com/Punyo/wordle}
\end{center}
また、Discordを用いて定期的に進捗報告や問題の共有を行った。
\section{機能}
プロジェクトで実装した主要機能を説明する。

\subsection{ゲームプレイ機能}
\begin{description}
  \item[ランダム出題]%
    \texttt{WordDAO.getRandomWord()} を用いて
    \texttt{Words} テーブルから1語を一様ランダムに選出し、
    \texttt{GameService} に渡してゲームを開始する。
  \item[入力受付]%
    キーボードからの直接入力と、GUIキーボードによる
    ローマ字入力の両方を受け付ける。
    ローマ字は \texttt{Latin2Hira} クラスでかな変換を行う。
  \item[判定ロジック]%
    \texttt{GameService.checkWord()} において
    入力単語と正解単語を 1 文字ずつ比較し,
    判定結果を列挙型 \texttt{WordBoxStatus}%
    (\texttt{CORRECT}/\texttt{IN\_WORD}/\texttt{NOT\_IN\_WORD})
    で返却する。
  \item[試行回数管理]%
    6 回まで推測をし,残りの試行回数は
    \texttt{GameService} 内部でデクリメントして管理する。
  \item[ゲーム終了処理]%
    正解時または試行回数消費時に
    \texttt{GameDialogController} を呼び出し,
    ダイアログを表示して結果を通知する。
\end{description}

\subsection{データ管理機能}
\begin{description}
  \item[記録保存]%
    各ゲーム終了時に
    \texttt{RecordDAO.saveRecord(RecordDTO)} を通じて
    \texttt{Records} テーブルに以下の情報を保存する。
    \begin{itemize}
      \item 出題単語 ID
      \item 試行回数
      \item クリア成否
      \item プレイ日時
    \end{itemize}
  \item[統計更新]%
    同一トランザクション内で
    \texttt{Words} テーブルの統計カラム
    (出題回数・クリア回数・失敗回数)を更新することで,
    データの整合性を確保する。
  \item[履歴取得]%
    \texttt{RecordDAO.getAllRecords()} を利用し,
    \texttt{RecordController} がテーブルビューに
    最新順で表示する。
\end{description}

\subsection{ユーザーインタフェース機能}
\begin{description}
  \item[メニュー画面]%
    \texttt{MenuController} により
    新規プレイ/履歴確認を選択できる。
  \item[プレイ画面]%
    \texttt{GameController} が
    \emph{単語グリッド} と \emph{仮想キーボード} を生成し,
    判定結果に応じてセルおよびキーボードの色を
    \texttt{CSS} クラスで動的に切り替える。
  \item[履歴画面]%
    \texttt{RecordController} が
    \texttt{Records} テーブルの内容を
    \texttt{TableView} に表示し,
    列ごとの並べ替えをサポートする。
  \item[ダイアログ]%
    \texttt{GameDialogController} で
    ゲーム結果・試行回数・正解単語を表示し,
    「もう一度プレイ」/「メニューへ戻る」の
    いずれかを選択できる。
\end{description}

\section{実装}
\subsection{DB設計}
本プロジェクトでは、SQLiteを使用してデータの永続化を実現している。データベースには以下の2つのテーブルが存在する。

\begin{figure}[h]
\centering
\includegraphics[width=0.8\textwidth]{er.png}
\caption{ER図}
\label{fig:er}
\end{figure}

\subsubsection{Wordsテーブル}
単語データを管理するテーブル。以下のカラムを持つ。
\begin{itemize}
  \item \texttt{id}:単語の一意識別子(INTEGER、PRIMARY KEY、AUTOINCREMENT)
  \item \texttt{word}:単語本体(TEXT、NOT NULL、UNIQUE、CHECK(length(word) = 5))
  \item \texttt{word\_normalized}:正規化された単語(TEXT、NOT NULL、UNIQUE、CHECK(length(word\_normalized) = 5))
  \item \texttt{playCount}:出題された回数(INTEGER、NOT NULL、DEFAULT 0)
  \item \texttt{clearCount}:正解された回数(INTEGER、NOT NULL、DEFAULT 0)
  \item \texttt{missCount}:失敗された回数(INTEGER、NOT NULL、DEFAULT 0)
\end{itemize}

\subsubsection{Recordsテーブル}
プレイ記録を管理するテーブル。以下のカラムを持つ。
\begin{itemize}
  \item \texttt{id}:記録の一意識別子(INTEGER、PRIMARY KEY、AUTOINCREMENT)
  \item \texttt{wordId}:出題された単語のID(INTEGER、NOT NULL、FOREIGN KEY)
  \item \texttt{answerCount}:回答回数(INTEGER、NOT NULL)
  \item \texttt{clear}:正解したかどうか(BOOLEAN、NOT NULL)
  \item \texttt{date}:プレイ日時(DATE、NOT NULL)
\end{itemize}

\subsubsection{インデックス}
検索の効率化のために、以下のインデックスを作成している。
\begin{itemize}
  \item \texttt{idx\_word\_normalized}:Wordsテーブルの\texttt{word\_normalized}カラムに対するインデックス
\end{itemize}

\subsubsection{制約}
データの整合性を保つために、以下の制約を設定している。
\begin{itemize}
  \item Wordsテーブル
  \begin{itemize}
    \item 単語は必ず5文字であること
    \item 単語と正規化された単語は一意であること
  \end{itemize}
  \item Recordsテーブル
  \begin{itemize}
    \item \texttt{wordId}はWordsテーブルの\texttt{id}を参照すること
  \end{itemize}
\end{itemize}

\subsection{com.programming.advanced.wordle.service}
このパッケージには主にゲームの進行管理や判定ロジック等が実装されている。以下のクラスが含まれる。
\begin{itemize}
  \item \texttt{GameService}:ゲームの状態(正解単語、残り試行回数、単語長など)を管理し、ゲームの開始や入力単語の判定処理を行うSingletonであるクラスである。
  \item \texttt{WordBoxStatus}:\texttt{GameService.checkWord(String inputWord)}での判定結果を表す列挙型。
  \item \texttt{GameServiceTest}:\texttt{GameService}のテストクラスで、JUnitを使用してゲームのロジックが正しく動作するかを確認する。
  \item \texttt{GameServiceNotInitializedTest}:\texttt{GameService}が初期化されていない状態でのテストクラスで、ゲームの初期化が行われていない場合の挙動を確認する。
\end{itemize}

それぞれのクラスの詳細を以下に示す。なお、getter/setterは省略する。
\subsubsection{GameService.java}

\begin{itemize}
  \item \texttt{getInstance()}:\texttt{GameService}のインスタンスを取得する。
  \item \texttt{startNewGame(String word, int attempts, int wordLength)}:新しいゲームを開始し、正解単語・試行回数・単語長をセットし、初期化フラグを立てる。
  \item \texttt{checkWord(String inputWord)}:入力単語を正解単語と比較し、各文字ごとに\texttt{WordBoxStatus}(正解・含まれる・含まれない)を判定して返す。試行回数も減少する。\\
  なお、引数の単語が正解の単語の長さと異なる場合、DBからのデータの読み取りに失敗した場合、試行回数が残っていない場合、\texttt{startNewGame} での初期化が行われていなかった場合は例外を投げる。さらに、入力語がDBに登録されていない場合はnullを返す。
\end{itemize}

\subsubsection{WordBoxStatus.java}

\begin{itemize}
  \item \texttt{CORRECT}:文字が正しい位置にある場合。
  \item \texttt{NOT\_IN\_WORD}:文字が単語に含まれない場合。
  \item \texttt{IN\_WORD}:文字が単語に含まれるが位置が違う場合。
\end{itemize}

\subsubsection{GameServiceTest.java}

\begin{itemize}
  \item \texttt{testCheckWordLength()}:入力単語の長さが正解単語と異なる場合に\texttt{IllegalArgumentException}が投げられることを確認するテスト。
  \item \texttt{testCheckWordCorrect()}:正解単語を入力した場合、すべての判定が\texttt{CORRECT}となることを確認するテスト。
  \item \texttt{testCheckWordStatus()}:一部の文字が正解、一部が不正解の場合の判定結果(\texttt{CORRECT}、\texttt{NOT\_IN\_WORD}、\texttt{IN\_WORD})が正しいことを確認するテスト。
  \item \texttt{testCheckWordNoRemainingAttempts()}:試行回数が0になった後に入力した場合、\texttt{IllegalStateException}が投げられることを確認するテスト。
  \item \texttt{testGetters()}:\texttt{GameService}のgetterメソッドが正しく値を返すことを確認するテスト。
\end{itemize}

\subsubsection{GameServiceNotInitializedTest.java}

\begin{itemize}
  \item \texttt{testCheckWordNotInitialized()}:\texttt{GameService}が初期化されていない状態で\texttt{checkWord}を呼び出した場合、\texttt{IllegalStateException}が投げられることを確認するテスト。
\end{itemize}

\subsection{com.programming.advanced.wordle.controller}
このパッケージには、ゲームの各画面を管理するコントローラクラスと、それに関連する機能が実装されている。以下のクラスが含まれる。

\subsubsection{GameController.java}
プレイ画面を管理するコントローラクラス。\texttt{BaseController.java}を継承し、ゲームの進行やユーザー入力の処理を行う。\\
なお、\texttt{updateCurrentRowCellsColor(WordBoxStatus[] status, int row)}と \\ \texttt{updateKeyboardColor(WordBoxStatus[] status)}には、Java 17で新たに採用された言語仕様であるswitchのパターンマッチングを使用している。これにより、判定結果に応じた処理を簡潔に記述できる。

\begin{itemize}
  \item \texttt{initialize()}:ゲームを初期化し、新しい単語を取得してグリッドとキーボードを構築する。
  \item \texttt{getCurrentRow()}:現在の試行行番号を取得する。
  \item \texttt{getCurrentCell()}:現在の文字入力位置に対応するセルを取得する。
  \item \texttt{setFocusOnCurrentCell()}:現在の入力セルにフォーカスを当てる。
  \item \texttt{putLetterOnCurrentCell(char letter)}:指定した文字を現在のセルに入力し、現在の単語に追加する。
  \item \texttt{setupWordGrid()}:ゲーム用のマス目(TextField)のグリッドを構築し、初期状態の入力可能セルを設定する。
  \item \texttt{createGridCell()}:文字マス(TextField)を生成し、入力された文字をひらがなに変換する処理やフォーカス制御を行う。
  \item \texttt{setupKeyboard()}:50音配列に従って仮想キーボードを構築する。また、BackspaceキーとEnterキーを追加する。
  \item \texttt{createActionKeyButton(String text)}:BackspaceやEnterなどのアクションキーを生成する。
  \item \texttt{handleEnter()}:現在の単語を\texttt{GameService}に渡して判定し、正解またはゲームオーバーならダイアログを表示する。そうでない場合は結果に応じてセルとキーボードの色を更新する。
  \item \texttt{createKeyboardButton(String text)}:文字キーのボタンを生成し、押下時に文字を入力するように設定する。
  \item \texttt{handleBackspace()}:入力中の単語から最後の文字を削除し、対応するセルの表示も削除する。
  \item \texttt{updateCurrentRowCellsColor(WordBoxStatus[] status, int row)}:指定行のセルに判定結果に応じた色を適用する。
  \item \texttt{updateKeyboardColor(WordBoxStatus[] status)}:現在の単語の各文字に対応するキーボードボタンの色を判定結果に基づいて更新する。
  \item \texttt{handleKeyPress(String letter)}:仮想キーボードからの文字入力に応じてセルに文字を入力する。
  \item \texttt{showGameDialog(boolean isClear)}:ゲームクリアまたは失敗時に結果ダイアログを表示し、記録を保存する。
\end{itemize}

\subsubsection{MenuController.java}
メニュー画面を管理するコントローラクラス。
\begin{itemize}
\item \texttt{handlePlayAction(ActionEvent e)}:プレイボタンが押されたときに呼び出され、ゲーム画面(\texttt{game})へ遷移する処理を行う。
\item \texttt{handleRecordAction(ActionEvent e)}:記録ボタンが押されたときに呼び出され、記録表示画面(\texttt{record})へ遷移する処理を行う。
\end{itemize}


\subsubsection{RecordController.java}
記録画面を管理するコントローラクラス。

\begin{itemize}
  \item \texttt{playTableView}:記録を表示するためのテーブルビュー。
  \item \texttt{dateColumn}:ゲームを行った日付を表示するカラム。
  \item \texttt{answerColumn}:ゲームの解答語を表示するカラム。
  \item \texttt{tryCountColumn}:解答までにかかった試行回数を表示するカラム。
  \item \texttt{isClearColumn}:ゲームをクリアしたかどうかのフラグを表示するカラム。「クリア」または「失敗」と表示されるようにカスタマイズされている。
  \item \texttt{initialize()}:
  \begin{itemize}
    \item 各カラムに対して、\texttt{PropertyValueFactory} を用いてデータのプロパティを紐づけている。
    \item \texttt{isClearColumn} に対してカスタムセルファクトリを設定し、クリア状態を日本語で表示。
    \item \texttt{RecordDAO} を利用してデータベースから全記録を取得し、\texttt{ObservableList} に変換してテーブルに設定。
    \item テーブルビューは編集不可・フォーカス不要に設定。
    \item 各カラムは並べ替え可能であり、再配置は不可に設定。
    \item テーブルメニューボタンの非表示化、および自動リサイズポリシーの設定。
  \end{itemize}
\end{itemize}

\subsubsection{GameDialogController.java}
ゲーム終了時に表示するダイアログ画面を管理するコントローラクラス。

\begin{itemize}
\item \texttt{initializeDialog(boolean isGameClear, int tryCount)}:ゲーム終了時の結果に応じて、\texttt{resultLabel} に「ゲームクリア」または「ゲームオーバー」と表示し、\texttt{tryCountLabel} に試行回数を表示する。
\item \texttt{handlePlayAgainButton(ActionEvent e)}:\texttt{GameService} の \texttt{resetGame()} を呼び出してゲーム状態を初期化し、現在のウィンドウを閉じてゲーム画面に遷移する。
\item \texttt{handleReturnToMenuButton(ActionEvent e)}:\texttt{GameService} の \texttt{resetGame()} を呼び出してゲーム状態を初期化し、現在のウィンドウを閉じてメニュー画面に遷移する。
\end{itemize}

\subsubsection{BaseController.java}
\texttt{GameController}と\texttt{MenuController}の親クラスであり、共通する機能を実装している。

\begin{itemize}
\item \texttt{handleReturnToMenuButton(ActionEvent e)}:メニュー画面に遷移する。共通の「戻る」ボタンの処理として、他のコントローラーで継承して使用される。
\end{itemize}

\subsection{com.programming.advanced.wordle.dao}
このパッケージには主にデータベースアクセスに関する実装が含まれている。以下のクラスが含まれる。
\begin{itemize}
  \item \texttt{WordDAO}:単語データの取得や検索を行うクラス。
  \item \texttt{RecordDAO}:プレイ記録の保存や統計情報の更新を行うクラス。
  \item \texttt{DatabaseInitializer}:データベースの初期化やテーブル作成を行うクラス。
\end{itemize}

それぞれのクラスの詳細を以下に示す。

\subsubsection{WordDAO.java}
\begin{itemize}
  \item \texttt{getRandomWord()}:データベースからランダムな単語を1つ取得する。
  \item \texttt{getWordById(int id)}:指定されたIDの単語を取得する。
  \item \texttt{getWordIdByWord(String word)}:指定された単語がデータベースに存在するか確認し、存在する場合はそのIDを返す。
\end{itemize}

\subsubsection{RecordDAO.java}
\begin{itemize}
  \item \texttt{saveRecord(RecordDTO dto)}:プレイ記録を保存し、単語の統計情報(プレイ回数、クリア回数、失敗回数)を更新する。トランザクション管理を行い、データの整合性を保証する。
\end{itemize}

\subsubsection{DatabaseInitializer.java}
\begin{itemize}
  \item \texttt{initializeDatabase()}:データベースの初期化を行う。SQLite JDBCドライバの登録、テーブルの作成、インデックスの作成を行う。
  \item \texttt{insertInitialWords(Connection conn)}:初期データとしてwords.txtから単語を読み込み、データベースに追加する。
\end{itemize}

\subsubsection{DatabaseTest.java}
\begin{itemize}
  \item \texttt{testDatabaseConnection()}:データベースの接続とテーブル構造を確認するテスト。以下の項目を検証する。
  \begin{itemize}
    \item 必要なテーブル(words、records)が存在すること
    \item 各テーブルのカラム定義が正しいこと
  \end{itemize}
  \item \texttt{testSaveRecordAndGetAllRecords()}:レコードの保存と取得をテストする。以下の項目を検証する。
  \begin{itemize}
    \item クリア記録の保存が成功すること
    \item 失敗記録の保存が成功すること
    \item 全レコードの取得が正しく行われること
    \item 最新のレコードが失敗記録であること
    \item 2番目のレコードがクリア記録であること
  \end{itemize}
  \item \texttt{testWordStatisticsUpdate()}:単語の統計情報の更新をテストする。以下の項目を検証する。
  \begin{itemize}
    \item クリア記録保存後の統計情報が正しく更新されること
    \item 失敗記録保存後の統計情報が正しく更新されること
    \item プレイ回数、クリア回数、失敗回数が期待通りに増加すること
  \end{itemize}
\end{itemize}

\subsection{単語データの作成}
\subsubsection{単語データの作成プロセス}
Wordsテーブルに挿入する初期データは、以下の手順で作成した。

\begin{enumerate}
  \item 語彙データの取得
  \begin{itemize}
    \item 松下語学学習ラボ(\url{http://www17408ui.sakura.ne.jp/tatsum/database.html})から語彙データを取得
    \item 取得したデータは、単語とその読み方、品詞情報を含むCSVファイル形式
  \end{itemize}
  
  \item 単語の抽出と変換
  \begin{itemize}
    \item 作成したPythonスクリプト(\texttt{extract\_five\_letter\_words.py})を使用
    \item CSVファイルから5文字の単語を抽出
    \item カタカナの読みをひらがなに変換
    \item 正規化処理を適用(小文字を大文字に変換など)
    \item 重複を除去
  \end{itemize}
  
  \item データベースへの登録
  \begin{itemize}
    \item 抽出した単語を\texttt{words.txt}として保存
    \item データベース初期化時に\texttt{words.txt}から単語を読み込み
    \item 現在のデータ数は2,090個
  \end{itemize}
\end{enumerate}

\subsubsection{extract\_five\_letter\_words.py}
このスクリプトは、データベースに登録する5文字の単語を抽出するために使用される。以下の機能を実装している。
\begin{itemize}
  \item \texttt{katakana\_to\_hiragana(katakana\_str)}:カタカナをひらがなに変換する関数。伸ばし棒(ー)はそのまま保持する。
  \item \texttt{normalize\_word(word)}:単語を正規化する関数。小文字を大文字に変換し、以下の変換を行う。
  \begin{itemize}
    \item ぁ→あ、ぃ→い、ぅ→う、ぇ→え、ぉ→お
    \item っ→つ
    \item ゃ→や、ゅ→ゆ、ょ→よ
  \end{itemize}
  \item \texttt{has\_diacritical\_marks(text)}:テキストに濁点・半濁点が含まれているかを判定する関数。
  \item \texttt{extract\_five\_letter\_words(input\_csv, output\_txt)}:CSVファイルから5文字の単語を抽出し、テキストファイルに出力する関数。以下の条件で単語を抽出する。
  \begin{itemize}
    \item 5文字の単語であること
    \item カタカナのみで構成されていること(伸ばし棒は許可)
    \item 品詞が「名詞-普通名詞-一般」であること
    \item 濁点・半濁点を含まないこと
    \item 正規化後に重複しないこと
  \end{itemize}
\end{itemize}


\subsection{GUIの作成}
JavaFXを使用して作成されたFXMLファイルについてプレビュー画像とともに説明する。それぞれのFXMLファイルは、対応するコントローラクラスと連携して画面のレイアウトや動作を定義している。

\subsubsection{game.fxml}

\begin{itemize}
  \item \texttt{returnToMenuButton}:メニュー画面に戻るためのボタン。\\
  クリック時に\texttt{handleReturnToMenuButton}メソッドが呼び出される。
  \item タイトル:画面上部に「WORDLE」というタイトルを表示する\texttt{Label}。
  \item \texttt{wordgrid}:単語入力用のグリッド。中央に配置され、各セル間に5pxの間隔が設定されている。
  \item \texttt{keyboard}:仮想キーボード用のグリッド。中央に配置され、各キー間に2pxの間隔が設定されている。
\end{itemize}

\begin{figure}[h]
\centering
\includegraphics[width=0.8\textwidth]{game_preview.png}
\caption{game.fxmlのプレビュー}
\label{fig:game_fxml}
\end{figure}

\subsubsection{menu.fxml}
\begin{itemize}
  \item タイトル:画面中央上部に「WORDLE」というタイトルを表示する\texttt{Label}。
  \item \texttt{playButton}:ゲームを開始するためのボタン。クリック時に\texttt{handlePlayAction}メソッドが呼び出される。
  \item \texttt{recordButton}:ゲームの記録を確認するためのボタン。クリック時に\texttt{handleRecordAction}メソッドが呼び出される。
\end{itemize}

\begin{figure}[h]
\centering
\includegraphics[width=0.8\textwidth]{menu_preview.png}
\caption{menu.fxmlのプレビュー}
\label{fig:menu_fxml}
\end{figure}

\subsubsection{record.fxml}

\begin{itemize}
  \item \texttt{returnToMenuButton}:メニュー画面に戻るためのボタン。クリック時に\texttt{handleReturnToMenuButton}メソッドが呼び出される。
  \item タイトル:画面上部に「遊んだ記録」というタイトルを表示する\texttt{Label}。
  \item \texttt{playTableView}:プレイ記録を表示するテーブルビュー。以下の列を持つ:
  \begin{itemize}
    \item \texttt{dateColumn}:プレイした日付を表示する列。
    \item \texttt{answerColumn}:正解の単語を表示する列。
    \item \texttt{tryCountColumn}:解答回数を表示する列。
    \item \texttt{isClearColumn}:ゲームの結果(クリア/失敗)を表示する列。
  \end{itemize}
\end{itemize}

\begin{figure}[h]
\centering
\includegraphics[width=0.8\textwidth]{record_preview.png}
\caption{record.fxmlのプレビュー}
\label{fig:record_fxml}
\end{figure}

\subsubsection{gameDialog.fxml}

\begin{itemize}
  \item \texttt{resultLabel}:ゲームの結果(クリアまたはゲームオーバー)を表示するラベル。
  \item \texttt{answerLabel}:正解の単語を表示するラベル。
  \item \texttt{tryCountLabel}:試行回数を表示するラベル。
  \item \texttt{playAgainButton}:「もう一度プレイする」ボタン。クリック時に\texttt{handlePlayAgainButton}メソッドが呼び出される。
  \item \texttt{returnToMenuButton}:「メニューに戻る」ボタン。クリック時に\texttt{handleReturnToMenuButton}メソッドが呼び出される。
\end{itemize}

\begin{figure}[h]
\centering
\includegraphics[width=0.8\textwidth]{gameDialog_preview.png}
\caption{gameDialog.fxmlのプレビュー}
\label{fig:gameDialog_fxml}
\end{figure}

\subsection{ローマ字からひらがなへの変換}
ゲームプレイ時に、ユーザーがキーボードからローマ字で入力した文字をひらがなに変換するためのクラスを実装した。
\subsubsection{Latin2Hira.java}
\begin{itemize}
\item \texttt{Latin2Hira()}:ローマ字とひらがなを対応付ける \texttt{HashMap} を初期化する。
清音・濁音(が/ぎ など)・半濁音(ぱ/ぴ など)、拗音(kya/sha など)、
長音記号「ー」を含む主要な組み合わせを登録するが、濁点付き文字は清音(か/さ など)にマッピングする。  
\item \texttt{latin2Hira(String s)}:ローマ字文字列 \texttt{s} をひらがなに変換して返す。  
      \begin{enumerate}
        \item 入力をすべて小文字化する。  
        \item 同一子音が連続する場合に促音「っ」を追加し、次の文字を処理する。  
        \item 3 文字→2 文字→1 文字の順で最長一致検索を行い、辞書に存在するローマ字シーケンスをひらがなへ変換して \texttt{StringBuilder} に追加する。  
        \item 変換結果を文字列として返す。  
      \end{enumerate}
\item \texttt{isConsonant(char c)}:引数 \texttt{c} が子音(a--z のうち母音を除く)かどうかを判定し、連続子音による促音「っ」の検出に利用する。  
\end{itemize}

\end{document}